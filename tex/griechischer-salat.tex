%!TEX root = ../Kochbuch.tex

% Complete recipe example
\begin{recipe}{Griechischer Salat}
    \graph{
        big=pic/griechischer-salat,
        small=pic/peperoni,
    }
    
    % \introduction{%
    %     Sauce? Sose? Sauce? Oder Sauße? Aber Eier!
    % }
    
    \ingredients{%
        3 Stück & Paprika\\
        400 g & Tomaten\\
        2 & Gurken\\
        400 g & Feta\\
        100 g & Oliven\\
        100 g & Zwiebeln
    }
    
    \preparation{%
        \step Mit das einfachste Rezept in diesem Heft...\\
        ~\\       
         
         Alles kleinschnibbeln, zusammenschmeißen und mit Essig, Öl, Salz und Pfeffer abschmecken. 
         
         Wäre doch alles im Leben so einfach. Und so gesund.

    }
    
    
    \setHeadlines{hinthead = ... vom Kochprofi}
    \hint{%
        Salat ist ungleich Salat! Quasi alle Lebensmittel, die es so gibt, findet man auch als Salat wieder: Tomaten-, Gurken-, und Maissalat, Wurstsalat vs. Fleischsalat (Mayonnaise ist der Unterschied!). Nudelsalat, Bulgur- oder Couscoussalat, Hirse-, Kartoffel-, Reis- und, wie könnte man ihn vergessen: Eiersalat!
        
        Als Hauptgericht nur teilweise empfehlenswert: Obstsalat.
    }
    
\end{recipe}