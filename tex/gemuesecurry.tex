%!TEX root = ../Kochbuch.tex

% Complete recipe example
\begin{recipe}{Gemüsecurry}
    \graph{
        small=pic/gemuesecurry,
        big=pic/gemuese
    }
    
    % \introduction{%
    %     Sauce? Sose? Sauce? Oder Sauße? Aber Eier!
    % }
    
    \ingredients{%
        4 Stück & Paprika\\
        400 g & Zucchini\\
        300 g & Karotten\\
        500 g & Tomaten\\
        400 g & Zwiebeln\\
        0,5 Knollen & Knoblauch\\
        400 g & Broccoli\\
        500 g & Ananas (Dose)\\
        800 g & Reis\\
        500 g & Kokosmilch
    }
    
    \preparation{%
        \step Als erstes könnt Ihr euch um den Reis kümmern, wir empfehlen dabei die \textbf{Quellreis-Methode}: Auf Jeweils einen Teil Reis kommen zwei Teile Wasser. Pro Liter Wasser könnt Ihr einen Teelöffel Salz hinzugeben, dann auf den Kocher stellen und die Kiste anschmeißen. Macht den Deckel drauf, dann geht’s schneller. 

        \step Das Gemüsecurry wird in Fachkreisen als "{}Reinschmeißgericht"{} bezeichnet, weil man alle Zutaten nach und nach dem Topf zuführt: 
        
        Als erstes schneidet Ihr die Zwiebeln klein und beginnt, sie anzubraten. Währenddessen können die anderen Sachen \emph{kleingeschnibbelt} und in dieser Reihenfolge verheizt werden: Karotten und Broccoli, Paprika und Zucchini, Ananas-Stückchen, Knoblauch und Tomaten.
        
        \step Wenn Ihr wollt, dass das Gemüse noch schön bissfest ist, dann bereitet schon alle Gemüsesorten vor, damit Ihr sie nur noch in den Topf geben müsst. So hat es nicht so viel Zeit, zu verkochen.
        
        \step Wenn alles im Topf ist, kann die Kokosmilch hinzugegeben werden. Manchmal setzt sich in der Dose das Fett der Kokosmilch als ein großer Klumpen ab, der gehört auch ins Essen. Hmmm, Klumpen!
        
        \step Jetzt könnt Ihr noch nach Belieben würzen. Die ganze Mixtur kocht jetzt wild vor sich hin und wartet nur auf eine Zusammenführung mit dem Reis in Euren Tellern. Wir hoffen, dass es mundet!
    }
    
    
    \setHeadlines{hinthead = Anni's Kniff}
    \hint{
    Oh man da wird ja ganz schön viel geschnibbelt. Am besten sammelt ihr allen Biomüll direkt in einer Schüssel (nicht in der Plastiktüte - ihh Plastik) und schickt fleißige Helfer damit zum Mülldepot. Was weg ist, ist weg.
    }
    
\end{recipe}