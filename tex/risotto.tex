%!TEX root = ../Kochbuch.tex

% Complete recipe example
\begin{recipe}{Risotto}
    \graph{
        small=pic/risotto,
        big=pic/risottoreis
    }
    
    \introduction{%
    Risotto ist DAS Gericht, das das Prädikat "{}schlotzig"{} trägt. Gemeint ist die Konsistenz, die der Reis erreichen soll, und die ist genau so, wie es das Wort vermuten lässt.
    }
    
    \ingredients{%
        1 kg & Risotto Reis\\
        800 g & Champignons\\
        500 g & Zwiebeln\\
        200 g & Butter\\
        2 Liter & Gemüsebrühe\\
        1 Bund & Petersilie\\
        300 g & Parmesan (gehobelt)
    }
    
    \preparation{%
        \step Als erstes müssen die Zwiebelchen klein geschnitten werden und in der Butter bei kleiner Flamme angeschwitzt werden. 
        
        \step Jetzt kommt der Coup (frz. für Kuh): Der Reis wird jetzt in den Topf gegeben und ein wenig angebraten, bis er überall schön glänzt. 
        
        \step Unter ständigem Rühren wird immer eine kleine Menge (ca 1-2 Fjellis) Gemüsebrühe dazugegeben. Wartet immer so lange, bis die Flüssigkeit ganz aufgesogen ist, bis Ihr das nächste Mal Brühe in den Topf schüttet. 
        
        \step Das Ganze wird so lange durchgeführt, bis der Reis schön schlotzig (bissfest-schleimig-cremig) ist. 
        
        \step Jetzt können die klein geschnittenen Champignons hinzugegeben werden, Paprika und Tomaten dazu, ein klein wenig weiter köcheln lassen und mit etwas Tomatenmark, Salz und Pfeffer abschmecken.
        
        \step Am Ende die Petersilie klein hacken und genau wie den Parmesan unterrühren. Fix servieren und die Stämme beneiden, die Käsespätzle essen... Zumindest Jonas und Kilian sind auf der Käseseite des Lebens, Anni würde immer Risotto wählen!
        
    }
    %
    % \hint{%
    %     In einigen Stämmen werden ähnliche Gerichte - gegebenenfalls auch ungekocht - unter dem Namen "{}Ingelschleim"{} angepriesen. Aus Gründen der Diskretion gehen wir hier aber nicht näher auf die betroffenen Gruppen ein.
    % }
    
\end{recipe}