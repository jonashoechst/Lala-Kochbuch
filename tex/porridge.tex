%!TEX root = ../Kochbuch.tex

% Complete recipe example
\begin{recipe}{Porridge}
    \graph{
        big=pic/porridge,
        small=pic/hafer
    }
    
    % \introduction{%
    % ``Das ist vergleichsweise einfach, aber man muss viel schneiden.''\flushright --- \emph{Jack the Ripper}
    % }
    
    \ingredients{%
        500 g & Haferflocken\\
        1,5 Liter & Milch\\
        50 g & Butter\\
        1 l & Wasser\\
        3 Stück & Bananen\\
        3 Stück & Äpfel\\
        3 Stück & Birnen\\
        500 g & Erdbeeren
    }
    
    \preparation{%
        \step Guten Morgen!
        ~\\
        
        \step Werft erstmal die Milch, das Wasser und die Butter in den Topf und kocht das Ganze auf - rühren, damit es nicht anbrennt!
        
        \step Kippt die Haferflocken dazu und schaltet das Gas aus - so kann nichts anbrennen. Kräftig umrühren und danach ziehen lassen.
        
        \step Währenddessen können die Früchte klein geschnitten werden. Und dann kann der ganze Schlonz serviert werden.
        
        \step Nach dem Essen könnt Ihr die Bombe platzen lassen: Porridge ist sonst auch unter dem unappetitlichen Namen „Haferschleim“ bekannt. Hmmmmmmmmmmmmmm...
        
    }
    
    \setHeadlines{hinthead = Insiderwissen}
    \hint{%
        In einigen Stämmen werden ähnliche Gerichte - gegebenenfalls auch ungekocht - unter dem Namen "{}Ingelschleim"{} angepriesen. Aus Gründen der Diskretion gehen wir hier aber nicht näher auf die betroffenen Gruppen ein.
    }
    
\end{recipe}