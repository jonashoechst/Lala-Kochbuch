%!TEX root = ../Kochbuch.tex

% Complete recipe example
\begin{recipe}{Spätzle mit Linsen}
    \graph{
        small=pic/spaetzle-linsen,
        big=pic/linsen
    }
    
    % \introduction{%
    % Risotto ist DAS Gericht, das das Prädikat "{}schlotzig"{} trägt. Gemeint ist die Konsistenz, die der Reis erreichen soll und die ist genauso, wie es das Wort vermuten lässt.
    % }
    
    \ingredients{%
       1,5 kg & Spätzle\\
       600 g & Linsen \emph{(trocken)}\\
       300 g & Zwiebeln\\
       300 g & Karotten\\
       3 Stange & Lauch\\
       0,5 Stück & Knollensellerie\\
       2 Liter & Gemüsebrühe
    }
    
    \preparation{%
        \step \textbf{ACHTUNG: Die Linsen sollen am besten am Abend vorher in warmes Wasser eingelegt werden, oder die Kochzeit verlängert sich um ca. 1 Stunde!}
        
        \step Die Linsen werden in einem Topf mit dem geschnittenen Lauch, Karotten und dem fein gewürfelten Sellerie in Gemüsebrühe gekocht. Gebt in den Topf auch geviertelte Zwiebeln hinzu, das gibt mehr Geschmäckle. 
        
        Und natürlich gilt: Deckel zu, das spart Energie!
        
        \step Parallel dazu können die Spätzle nach Anleitung zubereitet werden: Wasser zum Kochen bringen, Salz rein und die Spätzle hinterher. Ab und zu umrühren, und nach etwa 8 Minuten sollten sie fertig sein.
        
        \step Wenn beides fertig ist, nehmt die Zwiebeln aus der Linsenmasse heraus. Im Essenskreis wird beides gemeinsam auf die Teller geschmissen und alle können selbst mit Salz, Pfeffer und Weißweinessig nachwürzen. 
        
        Hmmm, good old Süddeutschland.
    }

    
    \setHeadlines{hinthead = Spätzle aus'm Ländle}
    \hint{%
        Zuhause kann man Spätzle ganz leicht selbst machen: Der Teig besteht aus Eiern, Mehl, Salz und Milch. Nur für die klassische Form braucht man eine Spätzlepresse oder ein Brett zum schaben. 
    }
    
\end{recipe}