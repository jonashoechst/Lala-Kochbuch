%!TEX root = ../Kochbuch.tex

% Complete recipe example
\begin{recipe}{Krautnudeln}
    \graph{
        big=pic/krautnudeln,
        small=pic/krautkopf
    }
    
    % \introduction{%
    % ``Das ist vergleichsweise einfach, aber man muss viel schneiden.''\flushright --- \emph{Jack the Ripper}
    % }
    
    \ingredients{%
        1,5 kg & Nudeln\\
        1 Kopf & Kohl\\
        500 g & Zwiebeln\\
        0,5 Knollen & Knoblauch
    }
    
    \preparation{%
        \step Als erstes werden die Zwiebeln klein geschnitten, gesalzen und in etwas Öl angeschwitzt.
        
        \step Während sie auf kleiner Stufe köcheln behandelt Ihr den Kohlkopf wie folgt:
Die Blätter, die außen liegen, werden abgemacht, der restliche Kopf wird geviertelt. Dann wird er mit einem scharfen Messer in feine Streifen geschnitten, so dass so genannte Juliennes \emph{(frz. für feine Streifen)} entstehen. Waschmaschine!
        
        Der Kohl wird, gemeinsam mit fein gehacktem Knoblauch, in den Topf mit den Zwiebeln getan. Deckel drauf, kleinste Stufe. Warten. 
        
        \step Währenddessen könnt Ihr die Nudeln kochen: Wasser zum kochen bringen, ein paar Löffel Salz rein und die Nudeln dazu.
        \step Die gekochten Nudeln zum Kraut geben, alles ordentlich durchmischen. Fertig!
        
        \step Im Essenskreis passt dazu Pfeffer und, auch wenn es komisch klingt, Zucker. Zum Portionieren des Zuckers empfiehlt sich ein kleiner Löffel. Denn ganz ehrlich, wir kennen doch unsere Kinders!?
        
    }
    
    % \hint{%
    %     Energiespartipp: Legt beim Kochen der Kartoffeln den Deckel auf den Topf und füllt diesen nur zu einem Drittel mit Wasser.
    % }
    
\end{recipe}