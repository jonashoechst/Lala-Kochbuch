\newpage
\section{Küchen-SOS - Was tun, wenn...}
\subsection*{... zu viel Salz im Essen gelandet ist?} % (fold)
\label{sub:_zu_viel_salz_im_essen_gelandet_ist}
\begin{itemize}
    \item Bei einer Suppe oder einem Eintopf könnt ihr einfach eine Kartoffel oder etwas Reis (am besten im Teebeutel) mit kochen und so das Salz wieder rausziehen. 
    \item Bei Soßen könnt ihr auch noch etwas Wasser hinzugießen und es einfach n bisschen länger einkochen lassen.
\end{itemize}
% subsection _zu_viel_salz_im_essen_gelandet_ist (end)

\subsection*{... das Essen unten anbrennt?} % (fold)
\label{sub:_das_essen_unten_anbrennt}
\begin{itemize}
    \item Einfach alles, was noch nicht verbrannt ist, in einen neuen Topf geben. Vorsichtig abheben.
\end{itemize}
% subsection _das_essen_unten_anbrennt (end)

\subsection*{... der Kräutergarten leer ist?} % (fold)
\label{sub:_der_krautergarten_leer_ist}
\begin{itemize}
    \item Das ist erstmal doof! Deswegen gilt: Nehmt bitte nur so viel, wie ihr auch braucht. Wenn ihr den kleinen Garten fleißig gießt und pflegt, sollte bald wieder etwas da sein. Bis dahin gibt es eben nur getrocknete Kräuter.
    \item Keine Panik: Die Kräuter für die Grüne Soße kriegt ihr von uns direkt.
\end{itemize}
% subsection _der_krautergarten_leer_ist (end)

\subsection*{... ihr das Gefühl habt, ihr kriegt zu wenig Fleisch zu den Hauptmahlzeiten?} % (fold)
\label{sub:_ihr_das_gefuhl_habt_ihr_kriegt_zu_wenig_fleisch_zu_den_hauptmahlzeiten}
\begin{itemize}
    \item Also erstmal: Wir haben uns ja was überlegt bei unserem Verpflegungskonzept und es ist garnicht gesund, jeden Tag Fleisch zu essen. Aber für die, die garnicht damit zurecht kommen: Man könnte ja einfach ein wenig Wurst von den Brotzeiten abzwacken und so die Hauptmahlzeiten verfeinern.
\end{itemize}
% subsection _ihr_das_gefuhl_habt_ihr_kriegt_zu_wenig_fleisch_zu_den_hauptmahlzeiten (end)

\subsection*{... wenn's brennt?} % (fold)
\label{sub:_wenn_s_brennt}
\begin{itemize}
    \item Ruhig bleiben und löschen – in heißes Fett kein Wasser kippen!!!!
\end{itemize}
% subsection _wenn_s_brennt (end)
